\documentclass[12pt,letterpaper]{article}
\usepackage[margin=1in]{geometry}

\usepackage{siunitx}

\usepackage[T1]{fontenc}
\usepackage{mathptmx}

\usepackage{hyperref}
\hypersetup{
  colorlinks,
  citecolor=black,
  filecolor=black,
  linkcolor=black,
  urlcolor=black
}

\title{HamSCI: Eclipse Simulator Documentation}
\author{Joshua S. Vega, WB2JSV\\ New Jersey Insitute of Technology}
\date{\textit{Last Updated: \today}}

\begin{document}
\begin{center}
  {\scshape\Large HamSCI: Eclipse Simulator Documentation\par}
  \vspace{1em}
  {\large Joshua S. Vega, WB2JSV\par}
  \vspace{1em}
  {\itshape\large Last Updated: \today\par}
\end{center}

\tableofcontents

\newpage

%
% INTRODUCTION
%
\section{Introduction}
\label{sec:introduction}

HamSCI's eclipse simulator (colloquially known as ``\texttt{eclipsesim}'') is a
software package developed by members of the HamSCI team at the New Jersey
Institute of Technology used to simulate HF (3-30 $\si{\mega\hertz}$) through
the ionosphere. Specifically, the package was developed in order to simulate the
effects of the August 21, 2017 total solar eclipse on HF amateur ``ham'' radio
communications. The bulk of the package is developed in MATLAB. However, the
inputs and outputs are both in the comma-separated values (CSV) format in order
to allow for easy interoperability with other analysis tools (such as those
written in Python, R, or Julia).

This document is written in order to provide the reader with an understand of
the inner workings of the simulation package as well as justify some of the
design decisions that were made during development. In addition, this document
is intended to be used as a helpful guide to using the package. Because of this,
as the package undergoes future changes, so too should this document in order to
maintain parity between the package and its documentation.

%
% DEPENDENCIES
%
\section{Dependencies}
\label{sec:dependencies}

The \texttt{eclipsesim} package relies on a few third-party dependencies in
order to implement some specialized functionality. This section provides a brief
overview of the different dependencies and their function.

\subsection{PHaRLAP}
\label{sec:dependencies:pharlap}

PHaRLAP\footnotemark is a robust HF raytracing toolkit developed by Dr. Manuel
Cervera of the Defence Science and Technology Group, Australia. It is used to
compute the actual ray paths of radio transmission from transmitter to
receiver. More information on how it is used is provided in
\autoref{sec:design}. Although the core of the toolkit is written in FORTRAN, a
MATLAB application programming interface (API) is provided. PHaRLAP also
provides a simple API for the IRI (see below) that is compatible with the ray
path computation API.

\footnotetext{\itshape``The results published in this paper were obtained using
  the HF propagation toolbox, PHaRLAP, created by Dr. Manuel Cervera, Defence
  Science and Technology Group, Australia
  (manuel.cervera@dsto.defence.gov.au). This toolbox is available by request
  from its author.''}

\subsection{IRI}
\label{sec:dependencies:iri}

The International Reference Ionosphere (IRI)\footnotemark is a ionospheric model
developed by the Committee on Space Research (COSPAR) and International Union of
Radio Science (URSI). It is periodically updated to incorporate new
measurements. It generates monthly averages for several ionospheric measurements
including electron density, and electron and ion temperatures.

\footnotetext{\url{http://irimodel.org/}}

\subsection{SAMI3}
\label{sec:dependencies:sami3}

%
% USAGE
%
\section{Usage}
\label{sec:usage}

The \texttt{eclipsesim} package provides two methods of execution
out-of-the-box: serially and in parallel. More information and specific
instructions on each execution method is discussed in \autoref{sec:usage:serial}
and \autoref{sec:usage:parallel}, respectively.

Regardless of the execution method used, \texttt{eclipsesim} is designed to be
executed on a Linux-based machine. While limited success has been achieved on
other operating systems (notably, Microsoft Windows), the execution scripts are
developed with Linux in mind and may not execute as expected in incompatible
environments. All instructions provided in this section are therefore intended
for Linux-based systems and may require modification for other operating
systems.

\subsection{Serial}
\label{sec:usage:serial}

\subsection{Parallel}
\label{sec:usage:parallel}

\subsection{Input}
\label{sec:usage:input}

\subsection{Output}
\label{sec:usage:output}

%
% DESIGN
%
\section{Design}
\label{sec:design}

\subsection{Rough Pass}
\label{sec:design:rough}

\subsection{Fine Pass}
\label{sec:design:fine}

\subsection{Plotting}
\label{sec:design:plotting}

%
% EXTENSIONS
%
\section{Extensions}
\label{sec:extensions}

\subsection{``10-hop''}
\label{sec:extensions:10hop}

%
% CONCLUSION
%
\section{Conclusion}
\label{sec:conclusion}

%
% REFERENCES
%
\section{References}
\label{sec:references}

\end{document}